\documentclass[11pt, letterpaper, oneside]{article}

% This document is a simple latex template for reports. It follows the official Lab9K Styleguide, https://github.com/lab9k/Styleguide.

% Packages

\usepackage[dutch]{babel}

\usepackage{fontspec}               % Use own font

\usepackage{geometry}               % Interface to change page dimensions
\usepackage{parskip}                % For better spacing and indenting of paragraphs

\usepackage{graphicx}               % Images
\usepackage[export]{adjustbox}				% Wrap text around figures
\usepackage{fancyhdr}               % Headers and footers
\usepackage{caption}                % Caption

\usepackage{sectsty}                % Manipulate fonts of various sectional headings
\usepackage{enumitem}               % Manipulate enumerate, mdwlist and paralist
\usepackage{makecell}               % Improved tabular layout

\usepackage{url}
\usepackage{hyperref}

\usepackage{anyfontsize}            % Specify font sizes
\usepackage{xcolor}                 % Define own colours

\usepackage{lipsum}                 % Lorem ipsum

%\usepackage{minted}				% For code snippets

% Configure content of title page

\title{Rapport \\ Chatbot Gentse Feesten 2018}
\author{}
\newcommand{\organisation}{Lab9K}
\newcommand{\promotor}{Hans Fraiponts}
\newcommand{\subject}{Rapport Chatbot Gentse Feesten 2018}
\date{23 juli 2018}

% Colour scheme (based on logo)

\definecolor{priColour}{HTML}{026495}              % primary colour
\definecolor{secColour}{HTML}{4a92b8}              % secondary colour

% General configuration and package setup

\geometry{}
\graphicspath{ {figuren/} }    % Map containing all images
\setlength{\headheight}{15pt}

\makeatletter \hypersetup{
	pdfauthor = {\@author},
	pdftitle = {\@title},
	pdfsubject = {\subject},
	colorlinks=true,
	linkcolor=priColour,
	filecolor=priColour,  
	urlcolor=priColour
}

% Fonts

\setmainfont{Ubuntu}
\setmonofont{Ubuntu}

% General styling

% Use the primary colour for all titles
\sectionfont{\color{priColour}}
\subsectionfont{\color{priColour}}
\subsubsectionfont{\color{priColour}}

% Header
\pagestyle{fancy}
\renewcommand{\headrulewidth}{0pt}
\rhead{\@date}

\begin{document}
	% TITEL %
	\begin{figure}
			\includegraphics[width=0.27\textwidth,keepaspectratio]{figuren/logo} % Logo of Lab9K
	\end{figure}
	\vspace*{0.35cm}
	
	\noindent
	\fontsize{30pt}{28pt}\selectfont\textcolor{priColour}{\textbf{\@title}}\newline
	
	\fontsize{11pt}{15pt}\selectfont
	
	% BODY %
	
	\section{Statistieken}
	
	Dagelijks maakten enkele tientallen gebruikers gebruik van de chatbot. In totaal waren er een goeie 250 gebruikers.

	\begin{figure}[h]
		\centering
		\includegraphics[width=0.7\linewidth]{./figuren/user-activity}
		\caption{Algemene User Activity}
		\label{user-activity}
	\end{figure}

	\newpage

	De user retention lag echter wel laag. Veel mensen maakten slechts éénmaal gebruik van de chatbot.
	
	\begin{figure}[h]
		\centering
		\includegraphics[width=0.7\linewidth]{./figuren/user-retention}
		\caption{User Retention}
		\label{user-retention}
	\end{figure}

	Tenslotte hieronder nog enkele verdere statistieken in verband met de user base.
	
	\begin{figure}[h]
		\centering
		\includegraphics[width=0.7\linewidth]{./figuren/people-metrics}
		\caption{People Metrics}
		\label{people-metrics}
	\end{figure}
	
\end{document}